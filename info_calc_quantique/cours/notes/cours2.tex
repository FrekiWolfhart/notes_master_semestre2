\section{Cours 2}
Il y a deux type de portes logiques:
\begin{itemize}
	\item les portes non-réversibles, comme la porte et
	\item les portes réversibles, comme la porte non
\end{itemize}

Une porte est dite réversible lorsqu'à partir de la sortie on peux reconstruire l'entrée sans ambigüité.

La porte de Toffoli est une porte logique contenant deux bits de contrôle. Elle est dit universelle, car, en faisant
varier les bits de contrôle, on peux avoir une très grande variété de portes logiques.

La porte de Fredkin est une porte à trois bits universelle. Elle comporte un bit de contrôle. Ses sorties sont:
\begin{itemize}
	\item Si x=0, la sortie de y est y, et la sortie de z est z.
	\item Si x=1, la sortie de y est z, et la sortie de z est y.
\end{itemize}

Une porte quantique est un opérateur unitaire agissant sur un ou plusieurs qubits.\\
Des exemples de portes quantiques sont:
\begin{itemize}
	\item la porte CNOT(Controlled NOT)
	\item la porte de Toffoli
	\item la porte de Fredkin
\end{itemize}

La linéarité rend la copie d'information impossible en informatique quantique.
