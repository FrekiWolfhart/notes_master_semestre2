\section{Cours 3}
On ne peux pas calculer un état bien défini, juste la probabilité de l'avoir en partant d'un état donné.

\subsection{Algorithme de Deutsch}
Cet algorithme définit la nature de la fonction f, c'est à dire si elle est constante ou pas.

\subsection{Algorithme de Grovel}
Il s'agit d'un algorithme de recherche. Cet algorithme consiste à trouver l'élément x$_0$ dans un tableau de N éléments
non triés en faisant en moyenne moins de $\frac{n}{2}$ tirages. Pour cet algorithme, l'état initial doit être une
superposition uniforme d'états. Pour faire cela, on crée une combinaison d'état avec des matrices d'Hadamard. Cet
algorithme va "signer" l'élément voulu, et on va ensuite l'extraire avec une porte inversant le signe des éléments non
nuls. Ensuite, on applique H$^{\otimes N}$ à l'entrée et à la sortie de chaque porte.
