\section{Introduction}
Dans l'informatique classique, l'unité de base est le bit. Dans l'informatique quantique, elle est nomée qubit, aussi
noté qbit.

\subsection{Espace d'Hilbert}
Un espace vectoriel normé sur $\mathbb{C}$, complet pour la distance issue de sa norme, dont la norme est un
vecteur x, ||x|| découle d'un produit scalaire, ou Hermitien, par la formule:$$||x||=\sqrt{<x,x>}$$
On utilise des espaces d'Hilbert de dimension 2 car on possède un système à deux niveaux. Les vecteurs de ce système
sont nommés \textbf{ket}.\\
Chaque espace d'Hilbert est noté sur une base orthonormée.\\
Comme tout vecteur de tout espace vectoriel, un vecteur générique $|\phi>$ admet une décomposition en fonction de |0> et
de |1>.

\subsection{Projecteurs}
Pour une base orthonormée arbitraire, on peut définir les projecteurs suivants: |0><0| et |1><1|.\\
Les projecteurs sont idempotents, c'est à dire, (|0><0|)$^2$ = |0><0|.

\subsection{Calcul de produit de Kronecker}
|0><0| = $\begin{pmatrix} 1\\ 0\\ \end{pmatrix}$ $\otimes$ (1 0) = 1.(1 0) et 0.(1 0) = $\begin{pmatrix} 1 & 0\\ 0 & 0\\
\end{pmatrix}$

\subsection{Notations}
\begin{itemize}
	\item <v|w>: produit scalaire
	\item |w><v|: produit externe de Kronecker
\end{itemize}
