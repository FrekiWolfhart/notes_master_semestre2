\section{Cours 4}
H$^2$=Identité.
\subsection{Algorithme de Shor}
Soit N un nombre composite tel que N=p$_1$p$_2$...p$_n$, où p$_i\in\mathbb{P}$.\\
Trouver un non-trivial de N, c'est à dire un nombre dont les diviseurs sont compris entre 1 et N.
\subsubsection{Partie classique}
Choisir un nombre aléatoire a tel que 1<a<N.\\
Calculer le GCD entre a et N. On peux utiliser l'algorithme d'Euclide, qui fait ça en temps polynomial.\\
Si GCD est différent de 1, cela veut dire que a est un facteur non-triviale de N.\\
Si GCD est égal à un, cela veut dire que N et a sont co-premiers. Dans ce cas, il faudra utiliser l'algorithme de Shor.

On veut donc calculer la fonction f$_{a/N}$(x)=a$^*$ModN.\\
Pour une grande majorité de "a", la période de f$_{a/N}$ sera un  nombre pair. Si jamais la période de a n'est pas 
paire, il faut choisir un autre a.\\
Si la période de a est paire, il faut soustraire 1 de chaque côté de l'égalité.
