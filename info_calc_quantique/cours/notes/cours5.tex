\section{Cours 5}
\subsection{Transformation de Fouriel quantique}
Changement de base entre la base canonique et la base de Fouriel.
\subsection{Protocole pour Shor}
On choisit aléatoirement un nombre a compris entre 1 et N. Si le GCD(a,N) est différent de 1, on applique Euclide.
Sinon, on applique Shor.\\
Avec Shor, on calcule la période de a, nommée r. Si elle est paire, on continue, sinon, on choisit un autre a
aléatoire.\\
On a donc x=a$^{n/2}$ModN, et donc, en ajoutant un +1 et un -1 à gauche et à droite, on peux donc calculer les GCD entre
les deux, c'est à dire x+1 et x-1, avec N, et on obtiendra les valeurs de p et de q.
