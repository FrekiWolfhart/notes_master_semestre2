\section{Introduction}
La cryptographie regroupe trois concepts de base:
\begin{enumerate}
	\item l'intégrité
	\item la confidentialité
	\item l'authentification
\end{enumerate}
Si le problème concerne un de ces trois points, alors c'est un problème cryptographique.

L'intégrité, c'est garantir qu'un message n'a subi aucune modification depuis son envoi. Cela est très utile lors de
téléchargement, et est primordial pour le commerce électronique.

Il y a deux types de confidentialité, celle des documents sur un support, et celle de communication sur un canal.\\
La confidentialité utilise un système de chiffrement pour que seul ceux authorisé puissent voir les données.\\
Le chiffrement de données date de la Rome antique, avec un système de chiffrement basique. on décale chaque lettre de
trois position vers la droite. Cette méthode de chiffrement s'appelle le \textbf{chiffrement par décalage}.\\
Il doit donc y avoir un nombre de clé \textbf{rédhibitoire}, c'est à dire trop élevé pour toutes les tester une à une.\\
En juin 2016, le supercalculateur Sunway TaihuLight était le plus puissant calculateur au monde, avec 10$^{17}$
opérations par seconde.\\
Comme norme de sécurité, l'ANSSI recommande des clés de 128 bits au minimum.

Et, enfin, il y a deux types d'authentification. L'authentification interactive, qui entre en jeu lors d'une
communication directe avec un interlocuteur, et la signature électronique, qui consiste à attacher à un message une
preuve non-interactive de son origine, ce qui est crucial pour le commercve électronique.

Les principes de Kerckhoff définissent les principes de base à un bon système de chiffrement.

Il y a deux types de chiffrements symétriques:
\begin{enumerate}
	\item celui par blocs, on découpe le texte en bloc
	\item celui par flots
\end{enumerate}

Différences cryptographie symétrique et assymétrique:
\begin{itemize}
	\item la cryptographie assymétrique crée 2$\times$ plus de clés que la symétrique
	\item la symétrique est 100 à 100$\times$ plus rapide que l'assymétrique
\end{itemize}

D'après Snowden, la NSA ne peux toujours pas casser le PGP, mais cette information peut être fausse maintenant.
Le PGP est un système hybride ente la cryptographie symétrique et assymétrique.

La \textbf{non-répudiation} est un concept de droit qui empêche de nier un contract. Elle consiste à prouver qu'un 
message a bien été émis par son expéditeur, ou qu'il a bien été reçu par son destinataire.
