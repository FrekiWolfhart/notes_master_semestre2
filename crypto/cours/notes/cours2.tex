\section{Cours 2}
\subsection{Cryptographie symétrique par flots}
Afin de faire un cryptage symétrique par flot, on peux faire un ou exclusif entre un message formé de n bits et une clé
secrète de même taille. Dans ce cas, le déchiffrement sera la même opération que le chiffrement. Il semblerait que ce
système fut utilisé par le Kremlin et la Maison Blanche.\\
\textbf{Il est important} de changer de clé entre chaques messages afin d'éviter les attaques à clairs connus.\\
Les chiffrements par flots peuvent être implémentés avec une mémoire réduite, et sont très adaptés à des moyens de
calculs, de mémoire, et de transmission contraints, tel que les téléphone, ou un usage militaire.

\subsubsection{Système RC4}
Le système RC4 fut inventé par l'un des inventeurs du RSA, Ronald Rivest. Il est utilisé par le WEP, le WPA, le cryptage
proposé par BitTorrent, les pdfs, et est aussi en option dans des systèmes comme SSH et SSL.\\
Il utilise un chiffrement de type Vernam, c'est à dire un ou exclusif comme expliqué plus haut.

\subsubsection{Registres à décalage linéaire (LFSR)}
Dans ce type de registre, on obtient la suite pseudo-aléatoire formant les bits de la clé longue à partir d'une relation
de récurrence linéaire. Le calcul du bit s$_{i+L}$ s'effectue facilement à partir d'un circuit synchrone muni de L
bascules D qui conservent les valeurs des L derniers bits produits.\\
Cependant, la structure linéaire de ce système le rend peu sûr pour une utilisation cryptographique.

\subsection{Cryptographie symétrique par blocs}
Dans un chiffrement par blocs, on coupe le texte en blocs, et on chiffre chaque bloc indépendamment. Les blocs font au
moins 128 bits.

\subsubsection{Bourrage de fichiers}
Le bourrage de fichiers consiste à ajouter des octets après le message clair. Le dernier octet du fichier inique le
nombre d'octets ajoutés.

\subsubsection{Modes opératoires}
Il y a différents modes opératoires:
\begin{itemize}
	\item ECB(Electronic CodeBook): on découpe les blocs et on les chiffre indépendamment, mode très naif.
	\item CBC(Cypher Bloc Chaining): on choisit un vecteur d'initialisation aléatoire, et on le stocke avec le bloc
	chiffré. Les blocs suivants utilisent le bloc précédent comme vecteur d'initialisation.
\end{itemize}
Il existe d'autres modes opératoires, certains évitant le bourrage.
