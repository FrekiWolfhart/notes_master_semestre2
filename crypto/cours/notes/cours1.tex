\section{Cours 1}
Une fonction de hachage sert à transformer un message de taille quelconque en un résumé court de taille constante.\\
Une fonction de hachage cryptographique doit garantir qu'une modification du message de base modifiera le résumé
correspondant.

La fonction de hachage cryptographique permet de s'assurer l'intégrité des fichiers téléchargés car il y a très peu de
chances que le fichier corrompu ai le même résumé que le fichier voulu.

Le chiffrement de mot de passe est nécessaire dans un système informatique qui demande un mot de passe pour y accéder.\\
Étant donné qu'on sait faire des collisions pour MD5, ce système de hachage doit être proscrit de nos jours.

La signature numérique d'un document est fabriquée à partir du chiffrage du message avec une clé privée.\\
Il est essentiel que la fonction de hachage utilisée dans une signature électronique sois résistante à la collision.\\
La collision est le fait que deux textes différents aient le même résumé. Une fonction de hachage n'ayant pas de
résistance à la collision ne peux pas être utilisée en cryptographie.
