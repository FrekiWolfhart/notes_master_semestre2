\section{Cours 4}
Théorème d'incomplétitude de Gödel: Il existe des problèmes mathématiques indémontrables.\\
Quand on parle des "mathématiques", on parle en général de la théorie des ensembles de Zermelo-Fraenkel.

\textbf{Remarque:} Les ensembles d'axiomes et de règles d'un système formel peuvent être infinis, mais ils doivent être
récursivement énumérables, ou semi-décidables.\\
\textbf{Définition:} Un système formel est dit \textbf{cohérent} si on ne peux pas démontrer un énoncé et sa négation. 
On le dit aussi \textbf{non contradictoire}.\\
\textbf{Définition:} Un système formel est \textbf{complet} si on peux démontrer tout les énoncés qui sont vrais.

\textbf{Théorème d'incomplétude 1:} Aucun système formel contenant l'arithmétique élémentaire ne peux être à la fois
\textbf{cohérent et complet}.\\
\textbf{Théroème d'incomplétude 2:} Aucun système formel contenant l'arithmétique élémentaire ne peux démontrer sa
propre cohérence.

\textbf{Définition:} Un système formel est \textbf{correct} si tout les énoncés qu'on peux prouver sont vrais.\\
On a donc correction$\Rightarrow$cohérence, mais cohérence$\not\Rightarrow$correction.\\
\textbf{Théorème:} Aucun système formel contenant l'arithmétique élémentaire ne peux être à la fois correct et complet.
