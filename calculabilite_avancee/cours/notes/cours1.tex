\section{Cours 1}
Questions/Problèmes du jeu de la vie:
\begin{enumerate}
	\item Garden of Eden: Une configuration sans antécédent. Question: Plus petit Garden of Eden dans le jeu de la
	vie?
	\item Forteress: Peux se défendre contre tout. Peux servir à protéger une autre construction. Question: Plus petite
	forteresse possible dans le Game of Life?
	\item Death Problem: Étant donné une config initial finie, est-ce que toutes les cellules vont mourir? (indécidable)
\end{enumerate}

\textbf{Définition:\\}
un automate cellulaire est défini par:
\begin{enumerate}
	\item La dimension de l'espace d
	\item Un ensemble d'états finis S
	\item Un voisinage N
	\item Une fonction locale f:S$^m\rightarrow$S
	\item Une configuration c:2$^d\rightarrow$S évolue en c' avec,
	$\forall$x$\in\mathbb{Z}^d$:c'(x)=f(c(x+n$_1$),...,c(x+n$_m$))
\end{enumerate}

Si les hypothèses physiques suivantes sont vraies:
\begin{enumerate}
	\item les lois de la physique sont homogènes dans l'espace
	\item les lois de la physique sont homogènes dans le temps 
	\item la vélocité de propagation de l'information est bornée
	\item la densité d'information est bornée
	\item il existe un etat quiescent
\end{enumerate}
alors, nous vivons dans un automate cellulaire.
