\section{Cours 1}
Afin de mieux le protéger contre des menaces, il est important de fairte un inventaire des composants du système
d'information. Il est aussi crucial de connaître les connections réseaux utilisées par les composants afin de pouvoir
sécuriser au maximum ces interactions, et les points de connection vers ces réseaux.

Afin de sécuriser un réseau interne, il est intéressant de le séparer en plusieurs sous-réseaux afin d'isoler les
éléments les plus sensibles, et d'en empêcher l'accès depuis des sources non controllées.

Il faut faire attention au matériel personnel, qui n'est pas forcément aussi bien maintenu ou sécurisé que le matériel
professionnel de l'établissement.

Sécurisation du wifi:
\begin{itemize}
	\item utiliser WPA2 et CCMP(Counter Cipher Mode Protocol)
	\item modifier le SSID du wifi
	\item changer les identifiants par défault de la box
	\item chiffrer les communication par clé
	\item ne surtout pas utiliser le WPS sans l'option qui le désactive après 5 tentatives de clé loupées
	\item Si vous devez utiliser un wifi public, utilisez un VPN.
\end{itemize}

Il faut toujours attribuer le minimum possible de droits et de privilège à chaque utilisateur.
