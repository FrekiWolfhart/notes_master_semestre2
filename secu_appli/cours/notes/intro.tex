\section{Introduction}
Un système d'information est composé d'actifs, c'est à dire d'objets à protéger. Quand on parle de sécuriser, on parle
de réduire les risques, et non pas de les annuler.

Les tentatives d'intrusion sur des système d'information ou des pc personnels ne sont pas nouveau, elles existent depuis
les années 1980.\\
Le but en général est l'acquisition de données pour les revendre, ou l'acquision de ressources pour les
mettre à dispositions pour d'autres attaques.

L'année 2020 a établit un nouveau record en terme de quantité de notifications d'attaques informatiques.

Il y a trois critères pour déterminer le niveau de sécurité requis:
\begin{enumerate}
	\item Disponibilité = accessibilité au moment voulu
	\item Intégrité = exactitude et complétude
	\item Confidentialité = accèssible que à ceux qui en ont besoin
\end{enumerate}

Il y a souvent un critère complémentaire associé à ces trois là, la Preuve, qui est équivalent à une confiance
suffisante. Il n'est pas obligatoire d'avoir un niveau de protection très fort sur chaque critère, il faut donc les
prioriser.

Il n'y a pas de distinction nette entre surêté et sécurité. Le but d'un expert de sécurité est donc de s'assurer que
les vulnérabilités sont maitrisées.
