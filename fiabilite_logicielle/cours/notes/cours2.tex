\section{Cours 2}
Comment définit-on correct? L'absence d'erreur à l'exécution? Le code a le comportement voulu?\\
On définit la correction du code et de la fonction par rapport à la spécification donnée.

\subsection{Spécification}
Il y a trois types de spécification:
\begin{enumerate}
	\item en langue naturelle
	\item en langage formel
	\item par formule logique
\end{enumerate}

\subsubsection{En langue naturelle}
Une spécification en langue naturelle est écrite dans un language utilisé tout les jours, comme le français ou
l'anglais.\\
Tout doit être écrit comme si la personne n'avait jamais rien fait d'autre que coder de sa vie, mais, attention, il ne
faut pas écrire comment on déduit le résultat de l'entrée. La personne qui code doit avoir un peu de job, quand même.

\subsubsection{En langage formel}
On décrit la spécification avec du vocabulaire formel, et en utilisant des symboles logiques et ou mathématiques, voir
en utilisant des automates pour représenter les interactions entre les différents composants du programme final voulu.

\subsubsection{Par formule logique}
On peux spécifier via plusieurs logiques:
\begin{itemize}
	\item La logique du premier ordre: cela permet d'écrire de manière propre des spécifications. Elle demande trois
	notions: la \textbf{précondition}, ce qui est requis avant l'exécution d'une fonction, la \textbf{postconditon}, ce 
	qu'on a après l'exécution de la fonction, et le \textbf{contrat}, qui détermine si les données en précondition sont 
	bonnes, et que le résultat en postcondition est bon si le calcul termine.
	\item La logique temporelle: souvent interprétée dans les systèmes ou il y a du temps discret, c'est à dire, des
	instants les uns après les autres.
	\item La logique d'ordre supérieur: on a pas eu le temps d'en parler.
\end{itemize}
