\section{Cours 1}
\subsection{Problème de concurrence}
\subsubsection{Exclusion mutuelle}
Deux preuves de correction possible:
\begin{enumerate}
	\item à la main: graphe
	\item via Spin: variable critique + assert
\end{enumerate}

Afin d'éviter les problèmes de Deadlock, on peux
faire une synchronisation par blocage.\\
Cela veut dire mettre une condition qui fait
attendre le processus jusqu'à ce qu'il puisse
accéder à la ressource.

On peux aussi utiliser des "sémaphores", via une
variable qui va incrémenter ou décrémenter selon
les appels faits. La valeur de la variable de
sémaphore doit être égale au nombre de processus
qui peuvent prendre la ressource en parallèle.

Il existe aussi l'algorithme de Fisher, qui est faux
si il n'y a pas de délais entre les instructions.
C'est un algorithme très efficace, mais a un
problème de contrainte de temps.

\subsection{Model-Checking via logique LTL}
Il s'agit de la logique permettant la vérification
via Spin sans utiliser d'assert.\\
Avec cette technique, on va travailler sur les
systèmes de transition.

La définition formelle de la logique LTL ressemble
assez à celle d'un automate, sauf que les états
sont étiquettés par les variables vraies à cet
état.\\
Une exécution est donc une suite supposée infinie.

Les opérateurs de la LTL:
\begin{itemize}
	\item $\lnot$: négation
	\item $\wedge$: et logique
	\item $\vee$: ou logique ($\lnot\wedge$)
	\item X$_\varphi$: $\varphi$ est vrai à
	l'instant suivant
	\item $\varphi$U$\psi$: $\varphi$ est vrai
	jusqu'à ce que $\psi$ soit vrai.
	\item $\Box$: globalement
	\item $\Diamond$: un jour
	\item $\varphi_1$R$\varphi_2$ $\varphi_2$
	toujours vraie jusqu'à $\varphi_1$ vraie
	(les deux peuvent l'être en parallèle)
\end{itemize}

T est un modèle de $\varphi$ ss toute exécution de T
valide $\varphi$.

Propriétés classiques:
\begin{itemize}
	\item Safety: $\Box_\varphi$
	\item Liveness: $\Diamond_\varphi$
	\item $\Diamond\Box_\varphi$: à partir d'un
	moment, $\varphi$ sera toujours vraie.
	\item $\Box\Diamond_\varphi$: $\varphi$ sera
	infiniment souvent vraie.
\end{itemize}

Une formule de LTL correspond à un
\textbf{automate de Buchi}.

\subsubsection{Automate de Buchi}
Au lieu de montrer que le programme vérifie
$\varphi$, on va montrer que le programme ne peux
pas vérifier $\lnot\varphi$, et on associe à
$\lnot\varphi$ un automate de Buchi.

On fait donc un produit carthésien entre le
programme et l'automate de $\lnot\varphi$, ce qui
fait un autre automate de Buchi.

Comment on associe un automate de Buchi à une
formule?

Il faut donner à l'automate un état final par
lequel on passe infiniment souvent. L'automate n'est
pas obligatoirement fini.\\
La lettre $\omega$ représente une suite infinie du
symbole.

Les automates non-déterministes ont plus de
puissance que les automates déterministes.\\
On sait si un automate est "vide" si on ne peux pas
retourner vers son état final à partir du dit
état final.

Chaque formule LTL a un automate de Buchi qui la
reconnaît exactement.

Traduire un automate de Buchi alternant en automate 
de Buchi tranditionnel est de complexité exponentiel
le en la taille de la formule.\\
L'algorithme de détermination du "vide" a une
complexité linéaire en la taille de
T$\times$A$_{\lnot\varphi}$

\subsection{Spin}
Spin prends un programme Promela avec une propriété
LTL et donne un graphe d'exécution et un automate
de Buchi correspondant à la négation de la
formule.\\
Cet automate de Buchi est un \textbf{never claim}.
