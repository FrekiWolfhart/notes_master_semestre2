\section{Cours 2}
\subsection{Introduction}
Protocoles cryptographiques:
\begin{itemize}
	\item Programmation concurrente
	\item Utilisation de primitives cryptographiques
	\item Description courte
\end{itemize}
Exemple de description de protocole cryptographique: A$\rightarrow$B:aenc($\left< A,~K_{AB}\right>$,K$_B^{Pub}$) tel que:
\begin{itemize}
	\item A$\rightarrow$B: A envoie à B
	\item Après le :, message envoyé tel que:
	\begin{itemize}
		\item aenc: chiffrement assymétrique
		\item K$_{AB}$: clé symétrique entre A et B
		\item K$_B^{Pub}$: clé publique de B
	\end{itemize}
\end{itemize}

Ces protocoles sont utilisés dans divers domaines, tels que:
\begin{itemize}
	\item Les connections sécurisées
	\item Le commerce en ligne, où ils assurent la non-répudation, c'est à dire l'impossibilité de nier
	l'implication dans l'échange
	\item Votes électroniques et administration électronique
\end{itemize}
Des exemples de protocoles utilisés sont TLS ou Kerberos.

Le vote électronique est contesté pour des raisons de sécurité, mais un système est utilisé dans certaines
organisation, il s'agit de Belenios. Il s'agit d'un logiciel de vote électronique ou par correspondance, dont le
protocole cryptographique a des propriétés de sécurités prouvées.

Analyse de protocoles:
\begin{itemize}
	\item Analyse de failles logiques, qui sera le centre d'intérêt du cours
	\item analyse d'erreurs d'implémentations
	\begin{itemize}
		\item Souvent implémenté en C pour des raisons d'efficacité
		\item un protocole de quelques lignes peut être utilisé par des milliers d'individus
	\end{itemize}
\end{itemize}
Il est important que la vérification puisse s'automatiser.\\
Pur ce cours, on supposera les protocoles incassable, et donc parfaits.

\subsection{Cadre}
Tout les protocoles sont publiés et n'ont rien de secret, pour des raisons de sécurité.\\
Il y a deux types d'ataquants possibles:
\begin{enumerate}
	\item passif, qui est un observateur qui ne fait que écouter l'échange
	\item actif, qui, en plus d'écouter l'échange, va intervenir dedans et intercepter voir altérer les messages
\end{enumerate}

Vocabulaire:
\begin{itemize}
	\item enc=chiffrement
	\item dec=déchiffrement
	\item a=assymétrique
	\item s=symétrique
\end{itemize}
Le chiffrement assymétrique est beaucoup plus lent( 100$\times$), mais plus sécurisé que le chiffrement symétrique.

Il se peux que les fonctions de hachage soient utilisées hors de' la cryptographie, pour pouvoir gérer plus facilement
de gros objets. Par exemple, les graphes générés par Spin.

\subsection{Modélisation}
Il y a deux types de modèles:
\begin{enumerate}
	\item Le modèle formel, qui sera étudié durant ce cours
	\begin{itemize}
		\item Les messages sont des termes formels
		\item L'attaquant est un ensemble de règles de calculs algébriques
		\item La sécurité est donnée à partir des règles, et de leur capacité à déduire le secret
	\end{itemize}
	\item Le modèle calculatoire
	\begin{itemize}
		\item Les messages sont des suites de bits
		\item L'attaquant est une machine de Turing de capacité polynomiale
		\item La sécurité est vérifié si il y a très peu de chances de trouver la faiblesse
	\end{itemize}
\end{enumerate}

Chaque modèle a des avantages et des inconveniants:
\begin{itemize}
	\item Le modèle formel est simple et automatisable
	\item Le modèle calculatoire est difficile et à faire à la main
\end{itemize}
Dans certains cas, prouver la sécurité dans le modèle formel la prouve aussi dans le modèle calculatoire.

\subsubsection{Modèle Formel}
Description d'un protocole:
\begin{itemize}
	\item Session protocolaire
	\item Rôles A,B,...
	\item Règles d'envoi et de réception de messages
	\item Messages constantes, paires, chiffrés
\end{itemize}
Comment il est utilise:
\begin{itemize}
	\item Les rôles instanciés par des agents, aussi appelés principaux
	\item Les sessions jouées en parallèle, de manière asynchrone
	\item L'attaquant est modélisé par l'environnement
\end{itemize}

Les constantes sont:
\begin{itemize}
	\item Des clés K
	\item Des nonces N
	\item Des identités Id
	\item Des chaines de caractères représentant le texte
\end{itemize}

Les opérateurs de ce modèle sont:
\begin{itemize}
	\item Chiffrement symétrique senc(x,y): x chiffré par y
	\item Chiffrement assymétrique aenc(x,y): x chiffré par y
	\item inv(K): clé inverse de la clé K
	\item pk(K): clé publique associée à la clé privée K
\end{itemize}
Un agent du modèle peut être malhonnête.\\
On peux créer un protocole correcte dont l'implémentation n'est pas sûre.

