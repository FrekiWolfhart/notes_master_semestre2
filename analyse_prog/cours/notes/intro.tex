\section{Introduction}
Trois parties pour le cours:
\begin{enumerate}
	\item Concurrence
	\item Protocole cryptologique
	\item Reverse engineering/Fonction à l'exécution
	/Attaques
\end{enumerate}

Chaque partie aura des logiciels associés:
\begin{enumerate}
	\item Promela/Spin
	\item Proverif
	\item GDB/Radare2/Ghidra
\end{enumerate}

La vérification dans le concurrent est encore plus
compliqué que dans le séquentiel.

Un protocole est un algorithme.

Sur Promela, -> et ; sont équivalents, mais -> est
utilisé plus souvent après une condition.\\
Le mot clé "active" permet de dire que le processus
peux être lancé. Si on ne mets pas active, il faut
le lancer manuellement depuis init.\\
Il est conseillé, si on lance les processus dans
init, de mettre les run dans un "atomic". On peux
donner des arguments aux processus.\\
Les crochets définissent les tableaux.\\
L'entrelacement entre P et Q est noté $P||Q$.\\
"atomic" et "d\_step" sont des mots clés
similaires.

Spin permet de tester des propriétés comme:
\begin{itemize}
	\item Exclusion mutuelle: un seul à accès à la
	ressource à la fois.
	\item Deadlock: chaque processus attend une
	ressource qu'un autre utilise.
	\item Famine: un processus n'a \textbf{jamais}
	accès à la ressource.
\end{itemize}
